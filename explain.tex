\documentclass{article}%ドキュメントのクラスを指定するコマンド
\usepackage[utf8]{inputenc}%ドキュメントの文字エンコーディングを指定するためのパッケージ
\usepackage{graphicx}%画像を挿入するためのパッケージ
\usepackage{amsmath}%高度な数式を記述するためのパッケージ
\usepackage{amssymb}%数学記号を拡張するためのパッケージ
\usepackage{amsthm}%定理、補題、証明などの環境を定義するためのパッケージ
\usepackage{listings}%ソースコードを美しく表示するためのパッケージ
\usepackage{color}%テキストや図の色を変更するためのパッケージ
\usepackage{fancyvrb}%コードやテキストを「そのままの形式」で表示するためのパッケージ
\usepackage{fancyhdr}%ヘッダーやフッターをカスタマイズするためのパッケージ
\usepackage{lipsum}%ダミーテキスト(Lorem Ipsum)を生成するためのパッケージ
\usepackage{hyperref}%ドキュメント内にハイパーリンクを追加するためのパッケージ
\usepackage{geometry}%ページの余白やレイアウトをカスタマイズするためのパッケージ
\geometry{top=20mm,bottom=20mm,left=20mm,right=20mm}
\usepackage{titlesec}%セクション(章、節、項など)のタイトルのスタイルをカスタマイズするためのパッケージ

\title{Explain about MaximumQuiz}  %ここでは日本語は表示されなくなる
\author{Tealand and Co-pilot}
\first_date{2024/4/17}

\begin{document}

\maketitle

\section{Explain about the program}

This is a quiz site that asks questions about collaborative development and the web for first and second year students.


\section{Members}

Tealand\newline
matsushita-sora\newline
flying-flow\newline
NIshioka\newline
Ebichiri\newline
005man\newline

\section{How to use the program}

access http://quiz.game.teams.maximum.vc/

\section{Type of quiz}

1:GitHub\newline
2:C++\newline
3:C#\newline
4:UnityHub\newline
5:HTML\newline
6:CSS\newline
7:JavaScript\newline
8:FITEE\newline
9:AITEE\newline

\section{Future additions}
1:Ability to count the number of correct answers to a question and rank them accordingly\newline
2:Separate the quiz into "Language","IPA" and "Class" 
3:Introduce quiz of class("離散数学""微分積分基礎""線形代数基礎""確率統計""数理論理学""生物基礎")
4:add advertisement function
5:Q and A page
6:maintenance page}
\end{document}
