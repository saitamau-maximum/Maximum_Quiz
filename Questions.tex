\documentclass{article}%ドキュメントのクラスを指定するコマンド
\usepackage[utf8]{inputenc}%ドキュメントの文字エンコーディングを指定するためのパッケージ
\usepackage{graphicx}%画像を挿入するためのパッケージ
\usepackage{amsmath}%高度な数式を記述するためのパッケージ
\usepackage{amssymb}%数学記号を拡張するためのパッケージ
\usepackage{amsthm}%定理、補題、証明などの環境を定義するためのパッケージ
\usepackage{listings}%ソースコードを美しく表示するためのパッケージ
\usepackage{color}%テキストや図の色を変更するためのパッケージ
\usepackage{fancyvrb}%コードやテキストを「そのままの形式」で表示するためのパッケージ
\usepackage{fancyhdr}%ヘッダーやフッターをカスタマイズするためのパッケージ
\usepackage{lipsum}%ダミーテキスト(Lorem Ipsum)を生成するためのパッケージ
\usepackage{hyperref}%ドキュメント内にハイパーリンクを追加するためのパッケージ
\usepackage{geometry}%ページの余白やレイアウトをカスタマイズするためのパッケージ
\geometry{top=20mm,bottom=20mm,left=20mm,right=20mm}
\usepackage{titlesec}%セクション(章、節、項など)のタイトルのスタイルをカスタマイズするためのパッケージ

\Questions{Questions}  %ここでは日本語は表示されなくなる


C++でのコメントアウトは次のうちどれか

1. //コメント\newline
2. %コメント\newline
3. #コメント\newline
4. \#コメント\newline
 


\end{document}